\documentclass[10pt,a4paper]{article}
\usepackage{placeins}
\usepackage[utf8]{inputenc}
\usepackage[ruled]{algorithm2e}
\usepackage{fullpage}
\usepackage{graphicx}
\usepackage{float}
\usepackage[portuguese]{babel}
\usepackage[]{amsmath}
\restylefloat{figure}

\usepackage[adobe-utopia]{mathdesign}
\usepackage[T1]{fontenc}

% \usepackage{mdframed}

\DeclareGraphicsExtensions{.jpg,.pdf}

\numberwithin{equation}{section}

\title{Relatório - Trabalho Prático 2: Mips I4}
\author{
    Eugênio Pacceli
    \and
    Jonatas Cavalcante
    \and
    Lucas Augusto
    \and
    Samuel Oliveira
    \and
    Victor Pires Diniz
}

\begin{document}
\maketitle
\begin{center}
Organização de Computadores 2 - 2º Semestre de 2015
\end{center}

\section{Introdução}

O segundo trabalho prático do semestre envolve a transformação do processador MIPS com \emph{pipeline} implementado no trabalho anterior em um processador superescalar \textbf{I4}. Esse breve relatório pretende discutir os novos módulos implementados e as mudanças realizadas, entrando em detalhes sobre o funcionamento do processador e sobre as dificuldades encontradas no desenvolvimento.

\end{document}